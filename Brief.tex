\documentclass[version=last, Briefvorlage]{scrlttr2}
%
\setkomavar{subject}{Übersendung der Unterlagen für Lohnsteuer 2016, 2017, 2018}
%Deutsche Trennungen, Anführungsstriche und mehr: 
\usepackage{german, ngerman} 
%€-Zeichen 
\usepackage{textcomp} 
%
\begin{document}
%
\begin{letter}{%
Lohi Lohnsteuerhilfe Bayern e. V.. \\
z. Hd. Theresa Gruß \\
Am Alten Viehmarkt 3 \\
84028 Landshut \\
	}
	%
	\opening{Sehr geehrte Frau Gruß,}
	%
	Im Anhang erhalten sie meine ``gesammelten Werke''. Ich hoffe ich hab an alles gedacht. Ich habe für die entsprechenden Jahre 
	die jeweiligen Checklisten von Ihrer Homepage benutzt. Dazu habe ich falls es relevant war, diese Checkbox nummeriert und ein 
	Post-It mit der entsprechenden Nummer versehen, damit sie sehen können welcher Nachweis für welchen Themenbereich gedacht war.
	
	Ich hab auch nochmal zu jedem Jahr ein separates Deckblatt gemacht mit den wichtigsten Infos oder Erklärungen. 
	
	Was mir gestern Abend dann noch eingefallen ist, ich bin seit dem 12.05.2018 wieder verheiratet. Muss ich hier etwas beachten, im Hinblick auf eine 
	gemeinsame Veranlagung zum Beispiel? Werden hier noch weitere Informationen benötigt?
	
	Bei Rückfragen können Sie mich jederzeit telefonisch oder per Mail kontaktieren.
	%
	\closing{Freundliche Grüße}
	%
\end{letter}
%
\end{document}
%